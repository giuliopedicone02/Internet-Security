\addcontentsline{toc}{chapter}{Bibliografia} %per fare inserire il capitolo nella tabella dei contenuti

\begin{thebibliography}{9}

\bibitem{pseudonymizationGDPR}
  Data Privacy Manager,
  \textit{Pseudonymization according to the GDPR},
  [Online]. Available: \url{https://dataprivacymanager.net/pseudonymization-according-to-the-gdpr/}. 

\bibitem{ENISA2021}
  ENISA,
  \textit{Pseudonymisation Techniques and Best Practices},
  [Online]. Available: \url{https://www.enisa.europa.eu/publications/pseudonymisation-techniques-and-best-practices}.

\bibitem{Pseudonymization Defense}
  LinkedIn,
  \textit{How Can You Protect Against Re-identification Attacks},
  [Online]. Available: \url{  https://www.linkedin.com/advice/0/how-can-you-protect-against-re-identification-attacks}.

\end{thebibliography}


