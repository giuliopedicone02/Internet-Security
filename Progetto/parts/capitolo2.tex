\chapter{Il GDPR e la Pseudonimizzazione}


\section{Pseudonimizzazione nel GDPR}

L'European Union Agency for Cybersecurity (ENISA) lavora dal 2004 per rendere l'Europa sicura dal punto di vista cibernetico. ENISA collabora con l'Unione Europea, i suoi Stati membri, il settore privato e i cittadini europei per sviluppare consigli e raccomandazioni sulle buone pratiche in materia di sicurezza delle informazioni. Assiste gli Stati membri dell'UE nell'implementazione della legislazione europea pertinente e lavora per migliorare la resilienza delle infrastrutture e delle reti di informazione critiche in Europa. ENISA cerca di potenziare l'esperienza esistente negli Stati membri dell'UE supportando lo sviluppo di comunità transfrontaliere impegnate a migliorare la sicurezza delle reti e delle informazioni in tutta l'UE. Dal 2019, l'ENISA ha elaborato schemi di certificazione della cybersecurity.Lo scorso 3 dicembre, l'European Union Agency for Cybersecurity (ENISA, precedentemente denominata European Union Agency for Network and Information Security) ha pubblicato un importante documento sulla pseudonimizzazione dal titolo \textit{"Pseudonymisation techniques and best practices"} in cui vengono proposti alcuni possibili scenari di attacco a cui sono esposti i nostri dati e le migliori tecniche di difesa oggi in circolazione.

\newpage

\section{Definizione di Pseudonimizzazione nel GDPR}
Nell'Articolo 4(5) del GDPR, il processo di pseudonimizzazione è definito come:
\begin{quote}
“Il trattamento dei dati personali in modo tale che i dati personali non possano più essere attribuiti a un interessato specifico senza l'utilizzo di informazioni aggiuntive, a condizione che tali informazioni aggiuntive siano conservate separatamente e siano soggette a misure tecniche e organizzative atte a garantire che i dati personali non siano attribuiti a una persona fisica identificata o identificabile.”
\end{quote}

\subsection{Benefici della Pseudonimizzazione secondo il GDPR}
Se sei un Responsabile della Protezione dei Dati (DPO), puoi vedere l'attrattiva e i benefici della pseudonimizzazione. Permette di identificare i dati se necessario, ma li rende inaccessibili agli utenti non autorizzati e consente ai responsabili e agli incaricati del trattamento dei dati di ridurre il rischio di una potenziale violazione dei dati e proteggere i dati personali.

Il GDPR richiede di adottare tutte le misure tecniche e organizzative appropriate per proteggere i dati personali, e la pseudonimizzazione può essere un metodo appropriato se si desidera mantenere l'utilità dei dati.

\subsection{I Dati Pseudonimizzati sono ancora Dati Personali secondo il GDPR?}
Un pseudonimo è ancora considerato un dato personale secondo il GDPR poiché il processo è reversibile e, con una chiave appropriata, è possibile identificare l'individuo. Il Considerando 26 spiega:
\begin{quote}
“…i dati personali che hanno subito pseudonimizzazione, che potrebbero essere attribuiti a una persona fisica utilizzando informazioni aggiuntive, dovrebbero essere considerati informazioni su una persona fisica identificabile.”
\end{quote}

Inoltre, durante una violazione dei dati, una chiave di crittografia potrebbe essere esposta, mettendo a rischio anche i dati pseudonimizzati.

\section{I Dati Anonimizzati sono ancora considerati Dati Personali?}
Il GDPR si preoccupa solo del trattamento dei dati personali relativi a una persona fisica che consente l'identificazione di un individuo direttamente o indirettamente.

Se i dati sono anonimizzati in modo che gli individui non possano più essere identificati, il GDPR semplicemente non li considera più dati personali. Tuttavia, l'anonimizzazione dei dati può spesso distruggere il valore che i dati hanno per la tua organizzazione.