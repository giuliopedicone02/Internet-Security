\addcontentsline{toc}{chapter}{Bibliografia} %per fare inserire il capitolo nella tabella dei contenuti

\begin{thebibliography}{9}

\bibitem{pseudonymizationGDPR}
  Data Privacy Manager,
  \textit{Pseudonymization according to the GDPR},
  [Online]. Available: \url{https://dataprivacymanager.net/pseudonymization-according-to-the-gdpr/}. 

\bibitem{ENISA2021}
  ENISA,
  \textit{Pseudonymisation Techniques and Best Practices},
  [Online]. Available: \url{https://www.enisa.europa.eu/publications/pseudonymisation-techniques-and-best-practices}.

\bibitem{Pseudonymization Defense}
  LinkedIn,
  \textit{How Can You Protect Against Re-identification Attacks},
  [Online]. Available: \url{  https://www.linkedin.com/advice/0/how-can-you-protect-against-re-identification-attacks}.

\bibitem{Data re-identification}
  Wikipedia,
  \textit{Data Re-Identification},
  [Online]. Available: \url{https://en.wikipedia.org/wiki/Data_re-identification}.

\bibitem{Guidelines of Pseudonymization}
  NewSchool,
  \textit{Guidelines Anonymization and Pseudonymization},
  [Online]. Available: \url{https://ispo.newschool.edu/guidelines/anonymization-pseudonymization/}.

\bibitem{Privacy by Evidence}
  Research Gate,
  \textit{Develop Privacy Friendly Software},
  [Online]. Available: \url{https://www.researchgate.net/publication/336043425_Privacy_by_Evidence_A_Methodology_to_Develop_Privacy-Friendly_Software_Applicationsn}.

\bibitem{Kaggle Healthcare Dataset}
  Kaggle,
  \textit{Healthcare Dataset},
  [Online]. Available: \url{https://www.kaggle.com/datasets/prasad22/healthcare-dataset?resource=download}.
  
\end{thebibliography}