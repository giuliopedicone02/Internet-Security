\chapter*{Glossario}

\section*{Dati Personali}
Si riferisce a qualsiasi informazione relativa a una persona fisica identificata o identificabile (interessato); una persona fisica identificabile è colui che può essere identificato, direttamente o indirettamente, in particolare mediante un identificativo come nome, numero di identificazione, dati di localizzazione, identificativo online o uno o più fattori specifici della identità fisica, fisiologica, genetica, mentale, economica, culturale o sociale di quella persona (GDPR, art. 4(1)).

\section*{Responsabile del Trattamento}
Persona fisica o giuridica, autorità pubblica, agenzia o altro ente che, da solo o congiuntamente ad altri, determina le finalità e i mezzi del trattamento dei dati personali (GDPR, art. 4(7)).

\section*{Responsabile del Trattamento}
Persona fisica o giuridica, autorità pubblica, agenzia o altro ente che tratta dati personali per conto del titolare del trattamento (GDPR, art. 4(8)).

\section*{Pseudonimizzazione}
Il trattamento dei dati personali in modo tale che i dati personali non possano più essere attribuiti a un soggetto specifico senza l'uso di informazioni aggiuntive, a condizione che tali informazioni aggiuntive siano conservate separatamente e siano soggette a misure tecniche e organizzative per garantire che i dati personali non siano attribuibili a una persona fisica identificata o identificabile (GDPR, art. 4(5)).

\section*{Anonimizzazione}
Un processo mediante il quale i dati personali vengono alterati in modo irreversibile in modo che un soggetto non possa essere più identificato direttamente o indirettamente, né dal titolare del trattamento da solo né in collaborazione con altre parti (ISO/TS 25237:2017).

\section*{Identificativo}
Un valore che identifica un elemento all'interno di uno schema di identificazione. Un identificativo univoco è associato a un solo elemento.

\section*{Pseudonimo}
Conosciuto anche come criptonimo o semplicemente nym, è un pezzo di informazione associato a un identificativo di un individuo o a qualsiasi altro tipo di dato personale (ad esempio, dati di localizzazione). I pseudonimi possono avere diversi gradi di collegabilità agli identificativi originali.

\section*{Avversario}
Un ente che cerca di rompere la pseudonimizzazione e collegare un pseudonimo (o un dataset pseudonimizzato) al detentore del pseudonimo.

\section*{Attacco di Re-identificazione}
Un attacco alla pseudonimizzazione eseguito da un avversario che mira a ri-identificare il detentore di un pseudonimo.
