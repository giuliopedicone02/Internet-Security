\chapter{Principali meccanismi di difesa}

\section{Cos'è un attacco di re-identificazione?}

Gli attacchi di re-identificazione sono metodi per de-anonimizzare i dati utilizzando informazioni aggiuntive, come database esterni, metadati o analisi statistica, per inferire le identità dei soggetti dei dati. Gli attacchi di re-identificazione possono essere classificati in tre tipi principali:

\begin{itemize}
    \item \textbf{Attacchi di collegamento (linking attack):} Coinvolgono il match di dati anonimizzati con altre fonti di dati che contengono informazioni identificative, come nomi, indirizzi o numeri di telefono.
    \item \textbf{Attacchi di inferenza (inference attack):} Basati sull'uso di metodi statistici o di apprendimento automatico per dedurre informazioni sensibili dai dati anonimizzati, come genere, età o stato di salute.
    \item \textbf{Attacchi di ricostruzione:} I più avanzati, utilizzano dati anonimizzati da più fonti o interrogazioni per ricostruire i dati originali o una loro approssimazione.
\end{itemize}

\section{Come prevenire gli attacchi di collegamento?}

Utilizzare \textbf{tecniche robuste di anonimizzazione} per garantire che ogni record anonimizzato sia indistinguibile dagli altri e che ogni gruppo di record abbia una diversità sufficiente di valori sensibili.

\section{Come prevenire gli attacchi di inferenza?}

Utilizzare l'\textbf{iniezione di rumore} per aggiungere\textbf{ errori casuali o controllati ai dati}, riducendo così la loro precisione e utilità per gli attaccanti. Applicare la privacy differenziale per garantire che la presenza o l'assenza di un individuo nei dati non influenzi l'esito delle analisi. Limitare la \textbf{quantità} e la\textbf{ granularità dei dati condivisi} e utilizzare meccanismi di controllo degli accessi e crittografia.

\section{Come prevenire gli attacchi di ricostruzione?}

Utilizzare la computazione sicura multiparte per consentire a più parti di eseguire calcoli sui loro dati senza rivelarli tra loro o a terzi. Adottare la crittografia omomorfica per eseguire operazioni su dati crittografati senza decifrarli. Sfruttare il learning federato per apprendere da dati locali in modo distribuito senza centralizzarli.


