\chapter{Definizione di Pseudonimizzazione}


La pseudonimizzazione è una tecnica di trattamento dei dati che mira a proteggere la privacy degli individui sostituendo i dati identificativi con pseudonimi. In questo modo, le informazioni originali non possono essere attribuite a una specifica persona senza l'uso di ulteriori informazioni che sono conservate separatamente.

Questa tecnica è utile per ridurre i rischi associati al trattamento dei dati personali, in quanto limita la possibilità di identificare gli individui senza avere accesso alle informazioni aggiuntive necessarie per invertire il processo di pseudonimizzazione.

La pseudonimizzazione è un metodo che permette di sostituire i dati originali (ad esempio, un indirizzo e-mail o un nome) con un alias o pseudonimo. È un processo reversibile che de-identifica i dati, consentendo la re-identificazione in seguito, se necessario. Questa tecnica è altamente raccomandata dal Regolamento Generale sulla Protezione dei Dati (GDPR) come uno dei metodi di protezione dei dati.

\section{Utilizzo della Pseudonimizzazione nella Protezione dei Dati}
La pseudonimizzazione facilita il trattamento dei dati personali, riducendo il rischio di esposizione di dati sensibili a personale e dipendenti non autorizzati.

\subsection{Esempio di utilizzo}
Ad esempio, quando si inviano fogli Excel contenenti dati sensibili via e-mail. Sebbene il mittente e il destinatario delle e-mail siano autorizzati ad accedere a tali informazioni, il supporto IT ha anche accesso a quelle e-mail. Ora immagina che si trattasse di bonus per il top management o informazioni sui salari aziendali. Quando i dati sono pseudonimizzati, c'è meno possibilità di esporre dati personali, poiché i record dei dati diventano non identificabili, rimanendo comunque adatti per l'elaborazione e l'analisi dei dati.

\subsection{Cos'è un Pseudonimo?}
In questo contesto, un pseudonimo è un identificatore associato a un individuo. Proprio come gli scrittori usano pseudonimi per nascondere la loro identità e proteggere la loro privacy, gli pseudonimi vengono utilizzati per lo stesso scopo nella protezione dei dati. Un pseudonimo può essere un numero, una lettera, un carattere speciale o una qualsiasi combinazione di questi legati a un dato personale specifico o a un individuo, rendendo quindi i dati più sicuri da usare in un contesto aziendale.